\input{../../common/preambleconf.tex}
\title{Bash经验}
\author{刘宇辉 \texttt{wolfpythonlondon@gmail.com}}

\date{09-29-2012}

\begin{document}
\maketitle

Bash,全名Bourne-Again SHell,是继(sh,Bourne Shell)之后在开源界最流行的shell.
大部分的Linux发行版都会在预装bash,即使像Debian GNU/Linux这样坚持dfsg的发行版
把/bin/sh链接为dash.在FreeBSD和NetBSD中,可以方便的通过ports和pkgsrc获取bash.

\section{bash的特点}

Bash作为一个shell,功能很强大.特点如下:\\

\begin{itemize}
\item{作业控制(Job Control),前台和后台控制进程(fg)}
\item{别名(alias)和函数(function)}
\item{TAB命令补全功能(command-complete)}
\item{内置命令(built-in commands)}
\item{支持输入输出重定向(io-redirection)}
\item{命令扩展(expansion)}
\item{.命令}
\item{脚本基本控制结构和语句}
\item{支持管道}
\item{可配置性选项}
\item{......}
\end{itemize}
                

%\lstset{language=Bash,
        backgroundcolor=\color{yellow},
        numbers=left,
        breaklines,
        basicstyle=\ttfamily\small,
        keywordstyle=\color{red}\ttfamily,
        identifierstyle=\color{blue}\ttfamily,
        stringstyle=\ttfamily,
        commentstyle=\ttfamily,
        caption=\lstname,
        captionpos=b}





\begin{lstlisting}[title={简单例子},abovecaptionskip=0.5cm]

        %echo one two three | awk '{print $2}'
        two
\end{lstlisting}



\end{document}



