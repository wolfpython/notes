\input{../common/preambleconf.tex}
\title{AWK笔记}
\author{刘宇辉 \texttt{wolfpythonlondon@gmail.com}}

\date{09-26-2012}

\begin{document}
\maketitle

AWK是一门脚本语言,具体历史不再赘述。目前比较知名的三个实现是nawk,gawk和mawk。其中mawk号称是速度最快的,不过好久
没有更新了\footnote{http://invisible-island.net/mawk/}。这里面主要涉及到gawk(GNU/Linux, FreeBSD, NetBSD)和nawk(NetBSD,
FreeBSD).相比于sed,AWK更善于处理列,更精确的说是字段(field)。

\section{AWK的基本用法}

\input {../common/awkconf_nolineno.tex}


AWK代码是由单引号(`')和大括号(\{\})括起来的。下面是一个非常简单的例子:\\

\begin{lstlisting}[title={简单例子},abovecaptionskip=0.5cm]

        %echo one two three | awk '{print $2}'
        two
\end{lstlisting}



\end{document}



