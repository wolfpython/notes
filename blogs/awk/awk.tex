\documentclass[11pt]{article}
\usepackage{fontspec,hyperref}
\usepackage{listings}
\usepackage{color}
\usepackage[BoldFont,SlantFont,CJKchecksingle]{xeCJK}
%载入粗体,斜体,禁止单个汉字一行

%\setCJKmainfont{AR PL KaitiM GB}
\setCJKmainfont{AR PL SungtiL GB} %缺省汉字字体为简体
\setCJKfamilyfont{kai}{AR PL KaitiM GB}
\setCJKfamilyfont{hei}{WenQuanYi Zen Hei}

%\setmainfont{DejaVu Sans Mono:style=Book}
\XeTeXlinebreaklocale "zh"  %断行设置为中文格式
\XeTeXlinebreakskip=0pt plus 1pt minus 0.1pt
\addtolength{\textwidth}{4cm}


%\title{Effective Tcp/ip Programming阅读 \\[2ex] \large{------笔记}}
%\author{刘宇辉 \texttt{wolfpythonlondon@gmail.com}}
%\date{\today}
%
%\begin{document}
%\maketitle
%
%\section{Socket API}
%
%\lstset{language=C,
%        %backgroundcolor=\color{green},
%        numbers=left,
%        breaklines,
%        basicstyle=\ttfamily\small,
%        keywordstyle=\color{red}\ttfamily,
%        identifierstyle=\color{blue}\ttfamily,
%        stringstyle=\ttfamily,
%        commentstyle=\ttfamily,
%        caption=\lstname,
%        captionpos=b}
%
%\lstinputlisting[name=main.cpp,
% %       linerange={28-85},
% %       firstnumber=28,
%        stepnumber=2]{/home/wolf/public/notes/blogs/code/etip/simplec.c} 
%
%\end{document}

\title{AWK笔记}
\author{刘宇辉 \texttt{wolfpythonlondon@gmail.com}}

\date{09-26-2012}

\begin{document}
\maketitle

AWK是一门脚本语言,具体历史不再赘述。目前比较知名的三个实现是nawk,gawk和mawk。其中mawk号称是速度最快的,不过好久
没有更新了\footnote{http://invisible-island.net/mawk/}。这里面主要涉及到gawk(GNU/Linux, FreeBSD, NetBSD)和nawk(NetBSD,
FreeBSD).相比于sed,AWK更善于处理列,更精确的说是字段(field)。

\section{AWK的基本用法}

\lstset{language=Awk,
        backgroundcolor=\color{yellow},
        %numbers=left,
        breaklines,
        basicstyle=\ttfamily\small,
        keywordstyle=\color{red}\ttfamily,
        identifierstyle=\color{blue}\ttfamily,
        stringstyle=\ttfamily,
        commentstyle=\ttfamily,
        caption=\lstname,
        captionpos=b}





AWK代码是由单引号(`')和大括号(\{\})括起来的。下面是一个非常简单的例子:\\

\begin{lstlisting}[title={简单例子},abovecaptionskip=0.5cm]

        %echo one two three | awk '{print $2}'
        two
\end{lstlisting}



\end{document}



