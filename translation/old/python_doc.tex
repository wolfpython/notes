\documentclass{book}
\usepackage{color,CJKutf8}
\begin{document}

%\begin{algorithm}
\begin{CJK}{UTF8}{nsung}
\title{Python 教程}
\author{译者:wolf python london \texttt{wolfpythonlondon@gmail.com}}
\date{08/11/2010}
\maketitle


\section*{7.2  读写文件}

open()函数返回一个文件对象,在通常情况下传递给它两个参数:open(filename, mode). \\

f ={\color{red} open}{\color{blue} ('/tmp/workfile', 'w')}\\

第一个参数是包含文件名的字符串。第二个参数是包含一些字符的字符串,这些字符描述了文件被使用的方式。当文件只用来读时,mode就是'r',当只用来写的时候,就是'w'(在路径下的文件名相同的文件将被抹去),'a'则是打开文件追加内容,任何写入文件的内容都将自动被加在文件末尾。'r+'打开文件既可读也可写。mode参数是可选择的;'r'是缺省的打开文件模式。\\

通常情况下,文件都是以文本文件的方式打开,也就是,我们把字符串读出或者写入文件,这些文件是以特定的编码格式存储的(缺省的是UTF-8)。在mode后面的'b'表示以二进制模式打开文件:现在,数据是以字节对象读出或者写入文件。这个模式用于所有不包含文本的文件。\\

在文本模式下,缺省的动作是在读文件时把平台相关的换行符(unix 是$\backslash$n,windows是$\backslash$r$\backslash$t)转换为$\backslash$n,在写入文件事,把$\backslash$n转换为平台相关的换行符。这个“暗箱”操作对于文本文件是大有裨益的,但是对于二进制数据(比如在JPEG和EXE文件中)却是致命的。记住在读写这类文件时,使用二进制模式。\\


\subsection*{7.2.1  文件对象的方法}

这个部分的余下例子假定文件对象f已经被创建。\\

f.read(size),读取文件一些数据,并作为字符串或字节对象返回。size是一个可选择的数字参数。当size被忽略或者为负数时,文件整体都被读取并返回。如果文件大小是你的内存容量的两倍,你应该对此负责。否则,至多size字节被读取,返回。如果到达文件末尾,f.read()返回空字符串('').\\

f.{\color{red}read()}

'this is the entire file.$\backslash$n'

f.{\color{red}read()}

''\\
	
f.readline()从文件读取一整行,并且在字符串末尾留下一个换行符。



\end{CJK}
\end{document}
